% This  is teh concept paper on the architecture of SLS. 

\documentclass{sig-alternate}

\usepackage{graphicx}
\usepackage{cite}
\usepackage{epsfig}
\usepackage{booktabs}
\usepackage{multirow}
\usepackage{float}

% \usepackage[hyphens]{url}
% \usepackage{color}
% \usepackage{stmaryrd}
% \usepackage{latexsym}
% \usepackage{alltt}
% \usepackage{amssymb}
% \usepackage{bytefield}
% \usepackage{array}
% \usepackage{listings}
% \usepackage[small]{caption}
% \usepackage{subfigure}

\begin{document}

\title{SLS}
\numberofauthors{3}

\author{ }

\maketitle

\pagenumbering{arabic}

\input{abstract}

%\category{C.2.1}{Computer--Communication Networks}{Network Architecture and Design}
%%\category{C.2.3}{Computer--Communication Networks}{Network Operations}[network management, network monitoring]
%\category{C.2.5}{Computer--Communication Networks}{Local and Wide-Area Networks}[internet]
%
%\terms{Performance}
%%ACM general terms (choose 1 or more): Algorithms, Design, Documentation, Economics, Experimentation, Human Factors, Languages, Legal Aspects, Management, Measurement, Performance, Reliability, Security, Standardization, Theory, Verification

% Introduction

\input{background}

\input{components}

\input{experimental}

%Results

\input{conclusion}

\input{acknowledgements}

\bibliographystyle{unsrt}
\bibliography{sls}

\end{document}
